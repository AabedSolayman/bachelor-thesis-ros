%--------------------------------------------------------------------
%
%Mustervorlage fuer eine Aufgabe
%
%--------------------------------------------------------------------
%
%Ueberschreiben der automatisch erzeugten Aufgabennummer
%Die folgende Aufgabennummer ergibt sich aus dem Stand des
%Z�hlers + 1
%\setcounter{chapter}{0}
%
\chapter{Summary}\label{chap:summary}
%
%
%
In conclusion, this thesis presents a new approach for mobile robot localization and navigation. This approach is different than other used approaches as it uses only a laser scanner. The laser scanner is fixed in the room. The idea behind that is to scan the surrounding area and detect the pose of the robot based on the scan information. The software used in this thesis was developed under ROS, which has different advantages as code extensibility and reusability. The project contains separate modules, namely the motion control, the localization, the laser scanner and the robot. In order to localize the robot, three different methods have been tested. After comparing the three methods, the circular object detection method has been used due to its efficiency and accuracy. Finally, a control algorithm was used to let the robot move along a pre-defined line. In order to test the controller, different pre-defined paths were given as input to the controller. Given that this was only a preliminary attempt to use this approach, it is important to improve the controller algorithm to achieve better performance. Furthermore, it is recommended to simulate the whole system using simulation tools like Gazebo, which is usually used for robotics applications developed under ROS.

